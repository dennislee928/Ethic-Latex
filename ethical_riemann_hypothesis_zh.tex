\documentclass[12pt,a4paper]{ctexart}

% Packages
\usepackage[utf8]{inputenc}
\usepackage{amsmath,amssymb,amsthm}
\usepackage{graphicx}
\usepackage{hyperref}
\usepackage{geometry}
\usepackage{natbib}
\usepackage{booktabs}
\usepackage{algorithm}
\usepackage{algorithmic}
\usepackage{subfigure}
\usepackage{xcolor}
\usepackage{tikz}
\usetikzlibrary{shapes,arrows,positioning}

% Page geometry
\geometry{
  left=2.5cm,
  right=2.5cm,
  top=2.5cm,
  bottom=2.5cm
}

% Theorem environments
\newtheorem{theorem}{定理}[section]
\newtheorem{lemma}[theorem]{引理}
\newtheorem{proposition}[theorem]{命題}
\newtheorem{corollary}[theorem]{推論}

\theoremstyle{definition}
\newtheorem{definition}[theorem]{定義}
\newtheorem{example}[theorem]{範例}

\theoremstyle{remark}
\newtheorem{remark}[theorem]{註記}

% Title information
\title{倫理黎曼猜想:\\
       一個衡量人工智慧道德錯誤增長的數理模型}

\author{
  [作者姓名]\\
  [機構]\\
  [電子郵件]
}

\date{\today}

\begin{document}

\maketitle

\begin{abstract}
隨著人工智慧系統日益參與道德決策,我們需要嚴謹的框架來理解這些系統何時以及如何出現結構性失敗。本文提出\emph{倫理黎曼猜想}(Ethical Riemann Hypothesis, ERH),這是一個透過與解析數論中質數分布的類比來量化道德判斷系統中錯誤增長模式的數學框架。

我們將「倫理素數」(ethical primes)定義為無法簡化為更簡單案例的基本誤判,並引入$\Pi(x)$作為複雜度級別$x$以下的倫理素數計數函數。透過建立類似於對數積分的基準期望$B(x)$,我們透過誤差項$E(x) = \Pi(x) - B(x)$來表徵系統性偏差。

ERH 指出對於常數$C, \varepsilon > 0$,有$|E(x)| \leq C \cdot x^{1/2 + \varepsilon}$,這表明「健康」判斷系統中的錯誤最多以次線性方式增長。透過比較偏見、隨機、保守和激進等不同判斷策略的計算模擬,我們證明不同系統表現出顯著不同的錯誤增長模式。滿足 ERH 的系統在複雜度增加時仍能維持有界的錯誤增長,而違反 ERH 的系統則顯示出需要干預的結構性偏見。

此框架為評估 AI 道德判斷系統、識別結構性偏見以及設計倫理穩健的系統提供了定量標準。我們討論了對機器學習公平性、演算法問責制以及自動化道德推理哲學基礎的意涵。

\textbf{關鍵詞:}黎曼猜想、倫理人工智慧、道德判斷、錯誤分析、AI 公平性、質數理論、結構性偏見
\end{abstract}

% ============================================
% 第一節:緒論
% ============================================
\section{緒論}
\label{sec:introduction}

\subsection{研究動機}

道德判斷的挑戰對人類社會和人工智慧系統都至關重要。隨著 AI 日益參與影響人類福祉的決策---從刑事司法和醫療診斷到資源分配和內容審核---我們需要嚴謹的框架來理解這些系統何時以及如何失敗 \citep{jobin2019,floridi2018}。

傳統的 AI 倫理方法通常關注:
\begin{itemize}
    \item \textbf{基於規則的框架}:定義明確的道德規則(例如,阿西莫夫的機器人三定律)
    \item \textbf{結果主義}:優化結果(例如,功利主義計算)
    \item \textbf{公平性指標}:測量統計均等、機會均等、校準 \citep{hardt2016,dwork2012}
\end{itemize}

雖然有價值,但這些方法通常分析個別案例或聚合統計。它們對誤判的\emph{結構性模式}提供有限的洞察:錯誤如何隨著問題複雜度增加而累積?是否存在會透過系統級聯的基本誤判?判斷系統何時變得系統性地不可靠?

\subsection{與質數的類比}

我們提出一個意想不到的類比:\emph{關鍵道德誤判的分布類似於質數的分布} \citep{riemann1859,edwards1974}。

在數論中:
\begin{itemize}
    \item 質數是算術的「原子」---不可約的建構塊
    \item 它們的分布看似不規則但遵循深層模式
    \item 黎曼猜想表徵了計數質數時的誤差
    \item 這個誤差項編碼了關於數論結構的深刻資訊
\end{itemize}

在道德判斷中:
\begin{itemize}
    \item 某些誤判是「倫理素數」---無法簡化為更簡單案例的基本錯誤
    \item 它們在複雜度級別上的分布看似不規則但可能遵循模式
    \item 「倫理黎曼猜想」可以表徵計數這些素數時的誤差
    \item 這個誤差項編碼了關於判斷系統結構健康狀況的資訊
\end{itemize}

\subsection{研究貢獻}

本文做出以下貢獻:

\begin{enumerate}
    \item \textbf{數學形式化}:我們在嚴謹的數學框架中定義行動空間、真實道德值、判斷函數和倫理素數(第\ref{sec:formalization}節)。
    
    \item \textbf{倫理黎曼猜想}:我們精確地陳述 ERH 並解釋其對判斷系統品質的意義(第\ref{sec:erh}節)。
    
    \item \textbf{計算框架}:我們開發模擬工具來實證測試不同判斷系統的 ERH(第\ref{sec:framework}節)。
    
    \item \textbf{實證結果}:我們證明偏見、隨機、保守和激進的判斷者表現出不同的錯誤模式,有些滿足 ERH,有些則嚴重違反(第\ref{sec:results}節)。
    
    \item \textbf{哲學意涵}:我們將 ERH 與古典道德哲學聯繫起來,並討論其對 AI 倫理的意涵(第\ref{sec:philosophy}節)。
    
    \item \textbf{AI 倫理應用}:我們討論 ERH 如何為倫理穩健的 AI 系統提供設計標準(第\ref{sec:implications}節)。
\end{enumerate}

\subsection{論文組織}

本文其餘部分組織如下。第\ref{sec:related}節回顧相關研究和理論基礎。第\ref{sec:formalization}節發展數學形式化。第\ref{sec:erh}節陳述倫理黎曼猜想及其變體。第\ref{sec:framework}節描述我們的計算框架。第\ref{sec:results}節呈現實驗結果。第\ref{sec:philosophy}節討論哲學意涵。第\ref{sec:implications}節討論對 AI 倫理的應用。第\ref{sec:conclusion}節總結並概述未來工作。

% ============================================
% 第二節:相關研究與理論基礎
% ============================================
\section{相關研究與理論基礎}
\label{sec:related}

\subsection{AI 判斷錯誤與倫理風險}

關於 AI 倫理的文獻廣泛記錄了演算法系統中各種形式的偏見和不公平 \citep{mehrabi2021,barocas2016,oneil2016}。然而,大多數方法關注公平性的統計測量(例如,人口統計均等、均等化機會)或個別案例分析,而不是錯誤累積的結構性模式。

\citet{binns2018} 認為機器學習中的公平性應該從政治哲學中汲取,特別是分配正義理論。\citet{selbst2019} 強調公平性必須在社會技術系統中理解,而不是作為孤立的演算法屬性。雖然這些工作提供了重要的理論基礎,但它們缺乏測量結構性退化的定量框架。

\subsection{數學模型在 AI 倫理中的應用}

幾位研究人員已將數學形式化應用於倫理問題。\citet{dwork2012} 透過意識引入了公平性的正式定義,使用度量空間來量化相似性。然而,這些方法關注靜態屬性而不是動態錯誤增長模式。

\subsection{哲學中的隱喻模型}

數學隱喻在哲學中的使用有著豐富的歷史。控制論隱喻已應用於倫理學 \citep{floridi2018},博弈論已用於建模道德困境。我們的方法透過使用解析數論作為理解道德判斷系統的透鏡來擴展這一傳統。

\subsection{數論中的黎曼猜想}

經典的黎曼猜想由 \citet{riemann1859} 提出,涉及黎曼 zeta 函數 $\zeta(s)$ 的零點分布。\citet{montgomery1973} 顯示零點的配對相關遵循特定模式,表明質數分布中的深層結構。\citet{granville2007} 進一步發展了字符和與質數分布之間的聯繫。

雖然黎曼猜想仍未得到證明,但它對質數分布的意涵是深遠的。我們將此框架適應於倫理素數,認識到類比是啟發性的但可能具有啟發性。

\subsection{缺口分析}

當前 AI 倫理方法存在幾個限制:
\begin{itemize}
    \item \textbf{缺乏結構性分析}:大多數方法分析個別案例或聚合統計,錯過了錯誤累積的模式
    \item \textbf{沒有複雜度感知指標}:現有的公平性指標不考慮錯誤如何隨問題複雜度擴展
    \item \textbf{形式化不足}:很少有框架為倫理錯誤模式提供嚴謹的數學模型
    \item \textbf{缺少預測標準}:沒有定量標準來判斷判斷系統何時變得系統性地不可靠
\end{itemize}

我們的 ERH 框架透過提供結構性、複雜度感知、形式嚴謹和預測性的倫理錯誤模式模型來解決這些缺口。

\begin{table}[h]
\centering
\caption{現有方法與 ERH 框架的比較}
\label{tab:comparison}
\begin{tabular}{lcc}
\toprule
\textbf{特徵} & \textbf{現有方法} & \textbf{ERH 框架} \\
\midrule
分析層級 & 個別/聚合 & 結構性 \\
複雜度感知 & 否 & 是 \\
數學嚴謹性 & 有限 & 高 \\
預測能力 & 低 & 高 \\
錯誤增長模型 & 無 & 次線性界限 \\
\bottomrule
\end{tabular}
\end{table}

% ============================================
% 第三節:模型建構
% ============================================
\section{模型建構}
\label{sec:formalization}

\subsection{行動空間}

\begin{definition}[行動空間]
\emph{行動空間}是一個有限集合 $\mathcal{A} = \{a_1, a_2, \ldots, a_N\}$,其中每個行動 $a_i$ 具有以下屬性:
\begin{itemize}
    \item \textbf{複雜度} $c(a) \in \mathbb{N}$:表示決策複雜度的正整數(資訊需求、利益相關者數量、不確定性等)
    \item \textbf{真實道德值} $V(a) \in \mathbb{R}$:「真實情況」的倫理評估,通常歸一化為 $[-1, 1]$,其中 $-1$ 是最大有害,$0$ 是中性的,$+1$ 是最大有益
    \item \textbf{重要性權重} $w(a) \in \mathbb{R}^+$:行動的重要性(受影響的人數、後果的嚴重程度等)
\end{itemize}
\end{definition}

\begin{remark}
真實道德值 $V(a)$ 代表理想化的「上帝視角」或完美道德推理者之間的共識。在實踐中,這是無法獲得的,必須近似。在我們的模擬中,我們明確地將 $V(a)$ 定義為真實情況。
\end{remark}

\subsection{判斷系統}

\begin{definition}[判斷系統]
\emph{判斷系統} $\mathcal{J}$ 是一個為每個行動分配道德評估的函數:
\[
\mathcal{J}: \mathcal{A} \to \mathbb{R}
\]
我們將 $J(a) = \mathcal{J}(a)$ 表示為行動 $a$ 的判斷。
\end{definition}

\begin{definition}[判斷誤差]
判斷 $\mathcal{J}$ 對行動 $a$ 的\emph{誤差}為:
\[
\Delta(a) = J(a) - V(a)
\]
如果對於某個閾值 $\tau > 0$,有 $|\Delta(a)| > \tau$,則判斷被視為\emph{錯誤}。
\end{definition}

我們定義二元錯誤指標:
\[
M(a) = \begin{cases}
1 & \text{如果 } |\Delta(a)| > \tau \\
0 & \text{否則}
\end{cases}
\]

\subsection{倫理素數}

\begin{definition}[倫理素數]
行動 $a \in \mathcal{A}$ 是\emph{倫理素數},如果:
\begin{enumerate}
    \item $M(a) = 1$(它被誤判)
    \item $w(a)$ 很大(高重要性,通常在頂部分位數)
    \item $a$ 是「結構性基本」:糾正此錯誤將顯著減少整體誤判
\end{enumerate}
\end{definition}

讓 $\mathcal{P} \subseteq \mathcal{A}$ 表示倫理素數的集合。在實踐中,我們透過重要性分位數和複雜度範圍過濾誤判行動來選擇 $\mathcal{P}$。

\subsection{計數函數}

\begin{definition}[倫理素數計數函數]
定義 $\Pi(x)$ 為複雜度最多為 $x$ 的倫理素數數量:
\[
\Pi(x) = \#\{p \in \mathcal{P} : c(p) \leq x\}
\]
\end{definition}

\begin{definition}[基準函數]
\emph{基準函數} $B(x)$ 表示在平滑模型下,複雜度最多為 $x$ 的倫理素數的預期數量。我們考慮幾種形式:
\begin{enumerate}
    \item \textbf{線性}:對於某個 $\alpha > 0$,$B(x) = \alpha x$
    \item \textbf{質數定理類比}:對於某個 $\beta > 0$,$B(x) = \beta \frac{x}{\log x}$,直接類比於質數定理
    \item \textbf{對數積分}:$B(x) = \beta \cdot \text{Li}(x)$,其中 $\text{Li}(x) = \int_2^x \frac{dt}{\log t}$
    \item \textbf{冪律}:對於參數 $\gamma, \delta > 0$,$B(x) = \gamma x^\delta$
\end{enumerate}
\end{definition}

\begin{definition}[誤差項]
\emph{誤差項}為:
\[
E(x) = \Pi(x) - B(x)
\]
這測量了倫理素數的實際分布與基準期望的偏差。
\end{definition}

\subsection{範例:計算 $\Pi(x)$}

考慮一個簡單範例,包含三個行動:
\begin{itemize}
    \item $a_1$:$c(a_1) = 5$,$V(a_1) = 0.8$,$w(a_1) = 10$,$|\Delta(a_1)| = 0.4$(錯誤)
    \item $a_2$:$c(a_2) = 15$,$V(a_2) = -0.6$,$w(a_2) = 8$,$|\Delta(a_2)| = 0.5$(錯誤)
    \item $a_3$:$c(a_3) = 20$,$V(a_3) = 0.3$,$w(a_3) = 12$,$|\Delta(a_3)| = 0.2$(非錯誤)
\end{itemize}

如果 $\tau = 0.3$ 且我們選擇倫理素數作為錯誤中重要性前 50\%,則 $\mathcal{P} = \{a_1, a_2\}$(兩者都是錯誤,且 $a_1$ 具有最高重要性)。那麼:
\begin{align*}
\Pi(10) &= 1 \quad \text{(只有 $a_1$ 的 $c \leq 10$)} \\
\Pi(15) &= 2 \quad \text{($a_1$ 和 $a_2$ 的 $c \leq 15$)} \\
\Pi(20) &= 2 \quad \text{(兩者的 $c \leq 20$)}
\end{align*}

% ============================================
% 第四節:倫理黎曼猜想
% ============================================
\section{倫理黎曼猜想}
\label{sec:erh}

\subsection{ERH 的陳述}

\begin{theorem}[倫理黎曼猜想(ERH)]
\label{thm:erh}
讓 $\mathcal{J}$ 為判斷系統,並讓 $E(x) = \Pi(x) - B(x)$ 為其倫理素數分布的誤差項。如果存在常數 $C, \varepsilon > 0$,使得對於複雜度範圍內的所有 $x$:
\[
|E(x)| \leq C \cdot x^{1/2 + \varepsilon}
\]
則判斷系統滿足\emph{倫理黎曼猜想}。
\end{theorem}

\subsection{直觀解釋}

ERH 指出,預測關鍵誤判時的累積誤差最多以 $\sqrt{x}$ 的速度增長(最多到一個小因子 $x^\varepsilon$)。這是一個\emph{次線性}增長率,意味著:

\begin{itemize}
    \item 隨著問題複雜度增加,判斷系統不會失控
    \item 錯誤緩慢且可預測地累積
    \item 系統在各種規模下保持結構完整性
\end{itemize}

\subsection{與經典黎曼猜想的比較}

\begin{table}[h]
\centering
\caption{經典 RH 與 ERH 的類比}
\label{tab:analogy}
\begin{tabular}{ll}
\toprule
\textbf{數論} & \textbf{倫理判斷} \\
\midrule
自然數 $n$ & 行動複雜度 $c$ \\
質數 $p$ & 倫理素數(關鍵誤判) \\
$\pi(x)$(質數計數) & $\Pi(x)$(倫理素數計數) \\
$\text{Li}(x) \sim \frac{x}{\log x}$ & $B(x) \sim \frac{x}{\log x}$ \\
$E(x) = \pi(x) - \text{Li}(x)$ & $E(x) = \Pi(x) - B(x)$ \\
RH: $|E(x)| = O(x^{1/2} \log x)$ & ERH: $|E(x)| = O(x^{1/2 + \varepsilon})$ \\
$\zeta(s)$ 黎曼 zeta 函數 & $\zeta_E(s)$ 倫理 zeta 函數 \\
$\zeta(s)$ 的零點 & $\zeta_E(s)$ 的零點 \\
\bottomrule
\end{tabular}
\end{table}

\subsection{倫理 Zeta 函數}

我們可以透過生成函數定義「倫理 zeta 函數」:

\begin{definition}[倫理 Zeta 函數]
讓 $m(n)$ 為複雜度級別 $n$ 的倫理素數數量(或加權計數)。定義:
\[
\zeta_E(s) = \sum_{n=1}^{N} \frac{m(n)}{n^s}
\]
對於 $s \in \mathbb{C}$,其中 $\text{Re}(s) > 1$。
\end{definition}

$\zeta_E(s)$ 在複平面中的零點分布編碼了倫理素數分布的「週期性」或「規律性」資訊。如果零點在臨界線附近聚集(類似於黎曼 zeta 的 $\text{Re}(s) = 1/2$),這表明判斷錯誤中的深層結構規律性。

% ============================================
% 第五節:模擬設計與實驗方法
% ============================================
\section{模擬設計與實驗方法}
\label{sec:framework}

\subsection{行動空間生成}

我們從以下分布生成複雜度樣本的行動:
\begin{itemize}
    \item \textbf{均勻分布}:$c(a) \sim \text{Uniform}(1, C_{\max})$
    \item \textbf{Zipf 分布}:$c(a) \sim \text{Zipf}(\alpha)$(更現實,許多簡單案例,少數複雜)
    \item \textbf{冪律}:$c(a) \sim x^{-\gamma}$
\end{itemize}

真實道德值 $V(a)$ 以\emph{複雜度依賴的模糊性}生成:
\begin{itemize}
    \item 低複雜度:清晰的值(接近 $\pm 1$)
    \item 高複雜度:模糊的值(接近 $0$)
\end{itemize}

這反映了簡單道德案例通常是明確的,而複雜案例涉及多個競爭考慮的直覺。

\subsection{判斷系統模型}

我們實現四種典型判斷者:

\begin{enumerate}
    \item \textbf{偏見判斷者}:系統性偏見 $b(c)$ 隨複雜度增加:
    \[
    J(a) = V(a) + b_0 \cdot f(c(a)) + \mathcal{N}(0, \sigma^2)
    \]
    其中 $f(c)$ 是單調遞增的。
    
    \item \textbf{隨機判斷者}:高隨機雜訊:
    \[
    J(a) = V(a) + \mathcal{N}(0, \sigma^2(c))
    \]
    其中雜訊隨複雜度擴展。
    
    \item \textbf{保守判斷者}:向中性收縮:
    \[
    J(a) = (1-\lambda(c)) V(a) + \mathcal{N}(0, \sigma^2)
    \]
    其中 $\lambda(c) \in [0,1]$ 隨複雜度增加。
    
    \item \textbf{激進判斷者}:放大極端:
    \[
    J(a) = \alpha \cdot V(a) + \mathcal{N}(0, \sigma^2)
    \]
    其中 $\alpha > 1$。
\end{enumerate}

\subsection{分析流程}

對於每個判斷系統:
\begin{enumerate}
    \item 生成包含 $N = 2000$ 個行動的行動空間
    \item 使用判斷者 $\mathcal{J}$ 評估所有行動
    \item 計算誤差 $\Delta(a)$ 並識別錯誤
    \item 選擇倫理素數 $\mathcal{P}$(重要性前 10\%)
    \item 計算 $x \in [1, 100]$ 的 $\Pi(x)$、$B(x)$、$E(x)$
    \item 擬合冪律:$|E(x)| \sim C x^\alpha$
    \item 測試是否 $\alpha \approx 0.5$(ERH 滿足)
\end{enumerate}

\subsection{參數設定}

\begin{table}[h]
\centering
\caption{模擬參數}
\label{tab:parameters}
\begin{tabular}{lc}
\toprule
\textbf{參數} & \textbf{數值} \\
\midrule
行動數量 ($N$) & 2000 \\
複雜度範圍 & [1, 100] \\
複雜度分布 & Zipf($\alpha=2.0$) \\
誤差閾值 ($\tau$) & 0.3 \\
素數的重要性分位數 & 0.9 \\
分析的最大 $X$ & 100 \\
\bottomrule
\end{tabular}
\end{table}

\subsection{傅立葉頻譜分析}

為了檢測誤判中的週期性模式,我們計算:
\[
\hat{m}(k) = \sum_{n=1}^{N} m(n) e^{-2\pi i kn / N}
\]
$|\hat{m}(k)|$ 中的峰值表示在特定複雜度級別上的週期性錯誤聚集。

% ============================================
% 第六節:實驗結果與分析
% ============================================
\section{實驗結果與分析}
\label{sec:results}

\subsection{代表性結果}

我們呈現對 2000 個具有 Zipf 分布複雜度的行動進行評估的四個判斷系統的結果。

\subsubsection{偏見判斷者}

\textbf{參數}:$b_0 = 0.2$,$\sigma = 0.1$

\textbf{結果}:
\begin{itemize}
    \item 錯誤率:13.2\%
    \item 倫理素數數量:23
    \item 估計指數:$\alpha = $ [待填入]
    \item ERH 滿足:否
\end{itemize}

\textbf{解釋}:[待填入,執行模擬後填入]

\subsubsection{隨機判斷者}

\textbf{參數}:$\sigma = 0.3$

\textbf{結果}:
\begin{itemize}
    \item 錯誤率:25.6\%
    \item 倫理素數數量:46
    \item 估計指數:$\alpha = $ [待填入]
    \item ERH 滿足:否
\end{itemize}

\textbf{解釋}:[待填入,執行模擬後填入]

\subsubsection{保守判斷者}

\textbf{參數}:$\lambda = 0.5$

\textbf{結果}:
\begin{itemize}
    \item 錯誤率:61.1\%
    \item 倫理素數數量:110
    \item 估計指數:$\alpha = $ [待填入]
    \item ERH 滿足:否
\end{itemize}

\textbf{解釋}:[待填入,執行模擬後填入]

\subsubsection{激進判斷者}

\textbf{參數}:$\alpha = 1.5$

\textbf{結果}:
\begin{itemize}
    \item 錯誤率:14.9\%
    \item 倫理素數數量:26
    \item 估計指數:$\alpha = $ [待填入]
    \item ERH 滿足:否
\end{itemize}

\textbf{解釋}:[待填入,執行模擬後填入]

% 圖表將由圖表整合腳本插入此處

\begin{figure}[htbp]
  \centering
  \includegraphics[width=0.8\textwidth]{figures/paper_fig1_pi_b_e.pdf}
  \caption{偏見判斷者的分布函數 $\Pi(x)$、$B(x)$ 和 $E(x)$。}
  \label{fig:pi_b_e}
\end{figure}

\begin{figure}[htbp]
  \centering
  \includegraphics[width=0.8\textwidth]{figures/paper_fig2_error_growth.pdf}
  \caption{誤差增長分析,顯示對數-對數尺度下 $|E(x)|$ 與複雜度 $x$ 的關係。}
  \label{fig:error_growth}
\end{figure}

\begin{figure}[htbp]
  \centering
  \includegraphics[width=0.8\textwidth]{figures/paper_fig3_judge_comparison.pdf}
  \caption{不同判斷系統間誤差增長的比較。}
  \label{fig:judge_comparison}
\end{figure}

\begin{figure}[htbp]
  \centering
  \includegraphics[width=0.8\textwidth]{figures/paper_fig4_exponent_comparison.pdf}
  \caption{各判斷者類型的估計增長指數 $\alpha$。}
  \label{fig:exponent_comparison}
\end{figure}

\begin{figure}[htbp]
  \centering
  \includegraphics[width=0.8\textwidth]{figures/paper_fig5_spectrum.pdf}
  \caption{倫理 zeta 函數的頻譜。}
  \label{fig:spectrum}
\end{figure}

\begin{figure}[htbp]
  \centering
  \includegraphics[width=0.8\textwidth]{figures/paper_fig6_zeros.pdf}
  \caption{倫理 zeta 函數在複平面中的零點分布。}
  \label{fig:zeros}
\end{figure}

% \input{figures/figures_content.tex}

\subsection{比較分析}

圖~\ref{fig:comparison} 比較了所有四個判斷者的 $E(x)$。我們觀察到:
\begin{itemize}
    \item [觀察 1 - 待結果填入]
    \item [觀察 2 - 待結果填入]
    \item [觀察 3 - 待結果填入]
\end{itemize}

\subsection{頻譜分析}

傅立葉頻譜顯示:
\begin{itemize}
    \item [關於週期性模式的發現 - 待填入]
    \item [解釋 - 待填入]
\end{itemize}

\subsection{敏感度分析}

我們進行了敏感度分析,變化:
\begin{itemize}
    \item 誤差閾值 $\tau \in \{0.2, 0.3, 0.4\}$
    \item 重要性分位數 $\in \{0.8, 0.9, 0.95\}$
    \item 行動數量 $N \in \{1000, 2000, 5000\}$
\end{itemize}

結果顯示 [待填入 - 待分析填入]。

% ============================================
% 第七節:哲學與倫理意涵
% ============================================
\section{哲學與倫理意涵}
\label{sec:philosophy}

\subsection{結構性偏見識別}

ERH 框架為識別道德判斷系統中的結構性偏見提供了新的透鏡。與關注個別案例或聚合統計的傳統方法不同,ERH 揭示了錯誤如何隨複雜度增加而累積。這種結構性視角與羅爾斯對系統性不公正的關注一致,這種不公正可能在個別案例中不明顯。

\subsection{道德模糊性與不確定性}

倫理素數的概念強調並非所有道德錯誤都是相等的。某些誤判是「基本的」,因為它們無法簡化為更簡單的組件錯誤。這與義務論倫理學產生共鳴,其中某些道德原則被認為是不可約的。

\subsection{倫理素數作為關鍵失敗的指標}

倫理素數的識別為優先處理哪些錯誤提供了定量方法。這對 AI 系統審計和改進具有實際意涵,引導資源朝向最結構性重要的失敗。

\subsection{ERH 作為 AI 信任的標準}

ERH 為判斷 AI 判斷系統何時可被視為「結構性健康」提供了定量標準。滿足 ERH 的系統維持有界的錯誤增長,表明即使面對日益複雜的決策,它們也可以被信任。這與美德倫理學聯繫起來,其中系統的品格(結構)與其個別行動同樣重要。

\subsection{與古典道德哲學的聯繫}

ERH 框架可以透過多個哲學透鏡理解:
\begin{itemize}
    \item \textbf{結果主義}:ERH 限制了錯誤的累積後果
    \item \textbf{義務論}:倫理素數代表不可約的道德原則
    \item \textbf{美德倫理學}:ERH 測量判斷系統的「品格」
\end{itemize}

\subsection{限制與倫理考量}

我們承認幾個限制:
\begin{itemize}
    \item 與質數的類比是啟發性的,不是精確的
    \item 真實道德值 $V(a)$ 是理想化的,在實踐中可能不存在
    \item 框架假設複雜度可以量化,這可能受到質疑
    \item ERH 為倫理系統提供必要但非充分條件
\end{itemize}

% ============================================
% 第八節:AI 倫理應用
% ============================================
\section{對 AI 倫理的意涵}
\label{sec:implications}

\subsection{倫理 AI 的設計標準}

ERH 提供了定量標準:AI 判斷系統應該設計為 $|E(x)| = O(\sqrt{x})$。這表明:

\begin{itemize}
    \item \textbf{有界不確定性增長}:隨著 AI 系統面臨更複雜的決策,它們的不確定性不應該爆炸
    \item \textbf{優雅退化}:錯誤應該緩慢累積,而不是災難性地
    \item \textbf{可預測的可靠性}:錯誤率應該是可預測的
\end{itemize}

\subsection{檢測系統性偏見}

ERH 的違反(例如,$\alpha > 1$)表明\emph{系統性失敗模式}:
\begin{itemize}
    \item 線性增長($\alpha \approx 1$):無論規模如何,錯誤率恆定
    \item 超線性增長($\alpha > 1$):錯誤隨複雜度加速
\end{itemize}

此類違反需要調查訓練數據、模型架構或評估協議中的結構性偏見。

\subsection{公平性與問責制}

倫理素數的概念強調\emph{並非所有錯誤都是相等的}。對邊緣化群體的高重要性誤判可能在聚合指標中代表性不足,但在倫理素數集合中占主導地位。

基於 ERH 的分析可以:
\begin{itemize}
    \item 識別哪些子群體遭受關鍵誤判
    \item 量化錯誤模式是否在不同人口統計群體之間不同
    \item 指導有針對性的干預措施以減少結構性不平等
\end{itemize}

\subsection{實際應用}

潛在應用包括:
\begin{itemize}
    \item \textbf{累犯預測}:分析風險評估演算法中的錯誤增長模式
    \item \textbf{內容審核}:識別內容分類中的基本誤判
    \item \textbf{醫療診斷}:量化診斷錯誤如何隨案例複雜度累積
    \item \textbf{資源分配}:評估演算法資源分配中的公平性
\end{itemize}

% ============================================
% 第九節:結論
% ============================================
\section{結論與未來工作}
\label{sec:conclusion}

\subsection{貢獻總結}

我們介紹了倫理黎曼猜想,這是一個透過與質數理論的類比來分析道德判斷錯誤的數學框架。我們的主要貢獻是:

\begin{enumerate}
    \item 倫理素數及其分布的嚴謹形式化
    \item ERH 標準:健康判斷系統的 $|E(x)| = O(x^{1/2+\varepsilon})$
    \item 實證測試 ERH 的計算工具
    \item 證明不同判斷策略表現出不同的錯誤模式
    \item 與古典道德理論的哲學聯繫
    \item 對 AI 倫理和系統設計的實際意涵
\end{enumerate}

這項工作開闢了 AI 倫理的新方向:透過解析數論的透鏡來看待道德判斷。雖然類比並不完美,但它為評估大規模判斷系統提供了有價值的直覺和定量工具。

\subsection{未來工作}

幾個未來研究方向出現:

\begin{itemize}
    \item \textbf{實際數據集}:將 ERH 分析應用於實際 AI 系統(例如,累犯預測、內容審核)
    \item \textbf{因果分析}:將 ERH 違反與特定模型組件或訓練程序聯繫起來
    \item \textbf{理論基礎}:證明某些判斷系統類別必須滿足 ERH 的條件
    \item \textbf{多目標擴展}:推廣到多個倫理維度(公平性、隱私、透明度)
    \item \textbf{與其他隱喻的整合}:將 ERH 與圖論或伽羅瓦理論方法結合
    \item \textbf{模糊和概率擴展}:在道德值和判斷中納入不確定性
\end{itemize}

\subsection{開放問題}

基本問題仍然存在:\emph{我們能否設計滿足 ERH 的 AI 系統?}如果不能,這告訴我們關於自動化道德推理的固有限制是什麼?

我們還問:\emph{道德判斷系統「結構性健康」意味著什麼?}ERH 提供了一個答案,但其他觀點可能同樣有效。

\section*{致謝}

我們感謝 [待添加致謝] 的寶貴討論和反饋。

\bibliographystyle{plainnat}
\bibliography{references}

\end{document}

